%%%%%%%%%%%%%%%%%%%%%%%%%%%%%%%%%%%%%%%%%%%%%%%%%%%%%%%%%%%%%%%%%%
\usepackage{amsmath}
\usepackage{xpatch}
\makeatletter
\makeatother
\usepackage{amssymb, amsfonts, latexsym, fullpage}
% \usepackage{multirow, hhline} %% To build tables with different sized rows/double dividing lines
%%%% Packages taken from the ubcdown main_script.Rmd template
\usepackage{ragged2e}
\usepackage{setspace}
\usepackage{tocloft}
\usepackage{float} 
\floatplacement{figure}{H} 
\usepackage{wrapfig}
\usepackage{pdfpages}
\usepackage{mathtools}
\usepackage{bm} %For bold lowercase Greek letters
\usepackage{natbib} %For references
\usepackage{xcolor} %Allows us to set color for comments
\usepackage{caption} %using this package to control the font size in the table and figure captions
\captionsetup{font=small} % \captionsetup{font=footnotesize}
\usepackage[linesnumbered,ruled]{algorithm2e} %To write algorithms in a neat format
\usepackage{mathrsfs} %Let's us use \mathscr{} to produce script letters
\usepackage{graphicx} %% Allows specification of a directory
%\graphicspath{ {"/Users/danielhadley/Documents/SUMMER 2019/Research/Hadley/R Files/Images/"} }
\usepackage{subcaption} %% To allow for side-by-side images


% \usepackage{xypic} %% For drawing vines
% \usepackage{mathabx} %% For the \notdivides symbol
% \usepackage{xinttools} %% For making dynamic tables in latex
% \usepackage{amsthm} %% Trying to get theorems to start after a line break.
% % \usepackage{showlabels} %% Allows us to see the labels given to equation, figures, etc. for easy reference


%\linespread{1.6}
\newtheorem{theorem}{Theorem}
% \newtheorem*{theorem*}{Theorem}
\newtheorem{lemma}[theorem]{Lemma}
% \newtheorem*{lemma*}{Lemma}
\newtheorem{proposition}[theorem]{Proposition}
% \newtheorem*{proposition*}{Proposition}
\newtheorem{corollary}[theorem]{Corollary}
% \newtheorem*{corollary*}{Corollary}
\newtheorem{defn}{Definition}
\newtheorem{definition}{Definition}
% \newtheorem*{definition*}{Definition}
% \newtheorem*{example*}{Example}
\newtheorem{example}{Example}
\newtheorem{exercise}{Exercise}
\newtheorem{assumption}{Assumption}[section]

%%%%%%%%%%%%%%%%%%%%%%%%%%%%%%%%%%%%%%%%%%%%%%%%%%%%%%%%%%%%%%%%%%
%% Extra environment provides a shaded box
% \usepackage{xcolor}
% \usepackage{comment}
% \definecolor{shadecolor}{gray}{0.9}
% \specialcomment{extra}{\begin{shaded}}{\end{shaded}}


%%%%%%%%%%%%%%%%%%%%%%%%%%%%%%%%%%%%%%%%%%%%%%%%%%%%%%%%%%%%%%%%%%
%% New environment for small font matrix
% \newenvironment{smat}
%   {\left[\begin{smallmatrix}}
%   {\end{smallmatrix}\right]}
  
\newcommand{\smat}[1]{\brac{\begin{smallmatrix} #1 \end{smallmatrix}}}

%% New environment for box matrix
\newcommand{\bmat}[1]{\begin{bmatrix} #1 \end{bmatrix}}

%% New environment for parentheses matrix
\newcommand{\pmat}[1]{\begin{pmatrix} #1 \end{pmatrix}}

%% New environment to number one line of an align* environment
\newcommand\numberthis{\addtocounter{equation}{1}\tag{\theequation}}

%%%%%%%%%%%%%%%%%%%%%%%%%%%%%%%%%%%%%%%%%%%%%%%%%%%%%%%%%%%%%%%%%%
%% Equation Environment so I don't have to write out equation* all the time
\newcommand{\eeq}[2][]{\begin{equation} \label{Eq:#1} #2 \end{equation}}

%%%%%%%%%%%%%%%%%%%%%%%%%%%%%%%%%%%%%%%%%%%%%%%%%%%%%%%%%%%%%%%%%%
%% Equation Environment so I don't have to write out equation* all the time
\newcommand{\eqq}[2][]{\begin{equation*} #2 \end{equation*}}

%%%%%%%%%%%%%%%%%%%%%%%%%%%%%%%%%%%%%%%%%%%%%%%%%%%%%%%%%%%%%%%%%%
%% Number equation by their section
\numberwithin{equation}{section}

%%%%%%%%%%%%%%%%%%%%%%%%%%%%%%%%%%%%%%%%%%%%%%%%%%%%%%%%%%%%%%%%%%
%% Environment for importing R files as code and formatting options
\usepackage{listings} %For adding line numbers to
\lstset
{ %Formatting for code in appendix
    language=R,
    basicstyle=\footnotesize,
    numbers=left,
    stepnumber=1,
    showstringspaces=false,
    tabsize=1,
    breaklines=true,
    breakatwhitespace=false,
}

%%%%%%%%%%%%%%%%%%%%%%%%%%%%%%%%%%%%%%%%%%%%%%%%%%%%%%%%%%%%%%%%%%
%% Matrix environment shortcut
\def\bma{\begin{pmatrix}}
\def\ema{\end{pmatrix}}

%% Allows specification of a directory
\usepackage{graphicx}
%\graphicspath{ {"/Users/danielhadley/Documents/SUMMER 2019/Research/Hou-Thesis/"} }

\setcounter{tocdepth}{5}
\setcounter{secnumdepth}{5}

%%%%%%%%%%%%%%%%%%%%%%%%%%%%%%%%%%%%%%%%%%%%%%%%%%%%%%%%%%%%%%%%%%
%% Create shortcuts for the set of real numbers, natural numbers, rational numbers, and infinity
\renewcommand{\a}{\alpha}
\newcommand{\ab}{\vecl{a}}
\newcommand{\abm}{\bm{\alpha}}
\newcommand{\abs}[1]{\left\vert #1 \right\vert} % Absolute Value that take size operator in [\big]
\newcommand{\AC}{\mcal{A}}
\newcommand{\alpb}{\vecg{\alpha}}
\newcommand{\AS}{\mscr{A}} % Script A for sets of states in MCMC
\newcommand{\avg}[1]{\overline{#1}}
\renewcommand{\b}{\beta}
\newcommand{\Bb}{\vecl{B}}
\newcommand{\bb}{\vecl{b}}
\newcommand{\bbI}{\mathbbm{I}}
\newcommand{\bbone}{\mathbbm{1}}
\newcommand{\bbP}[1]{\mathbb{P}\paren{ #1 }} %% Probability as an operator
\newcommand{\BC}{\mcal{B}}
\newcommand{\bdry}[1]{\pd\!#1} %% Function to denote a boundary
\newcommand{\betab}{\vecg{\beta}}
\newcommand{\bi}{\Leftrightarrow} %%a biconditional arrow
\newcommand{\bias}[1]{\text{Bias}\paren{#1}}
\newcommand{\brac}[1]{\left[ #1 \right]} %% Brackets that dynamically sizene
\renewcommand{\C}{\mathbb{C}} %% Complex numbers
\newcommand{\cas}{\stackrel{as}{\to}} %% converges almost surely
\newcommand{\cb}{\mathbf{c}}
\newcommand{\CC}{\mcal{C}}
\newcommand{\ceil}[1]{\left\lceil #1 \right\rceil} %% Ceiling function with dynamic sizing
\newcommand{\cid}{\stackrel{d}{\to}} %% convergence in distribution symbol
\newcommand{\cip}{\stackrel{p}{\to}} %% convergence in probability symbol
\newcommand{\cond}{\ \middle| \ }
\newcommand{\Corr}[1]{\text{Corr}\!\left(#1\right)} %% write Corr as text in math mode for correlation operator
\newcommand{\Cov}[1]{\text{Cov}\!\left(#1\right)} %% write Cov as text in math mode for covariance operator
\newcommand{\D}{\Delta} %% Uppercase delta
\newcommand{\de}{\delta} %% Lowercase delta
\newcommand{\deriv}[1]{\frac{d}{d #1}} %% univariate derivative symbol
\newcommand{\Db}{\mathbf{D}} %% bold capital latin letter D
\newcommand{\DC}{{\cal D}}
\newcommand{\db}{\mathbf{d}} %% bold lowercase latin letter D
\newcommand{\diag}[1]{\text{diag}\paren{#1}}
\newcommand{\dist}[3][]{\fnc{\text{#2}_{#1}}{#3}}
\newcommand{\dom}[1]{\mathrm{dom}\paren{#1}} %% domain of a function
\newcommand{\dotb}{\bullet} %% slightly bigger cdot symbol for dot product
\newcommand{\Eb}{\vecl{E}}
\newcommand{\eb}{\vecl{e}}
\newcommand{\EC}{{\cal E}}
\newcommand{\eid}{\stackrel{d}{=}} %% equal in distribution
\newcommand{\ep}{\epsilon}
\newcommand{\epb}{\vecg{\epsilon}}
\newcommand{\epp}{\varepsilon}
\newcommand{\eppbm}{\bm{\varepsilon}} %% bold lowercase epsilon variation
\newcommand{\eqstar}{\stackrel{*}{=}} %% equal sign with a star over it for a footnote
\newcommand{\errtr}{\text{Err}_{\text{tr}}}
\newcommand{\errin}{\text{Err}_{\text{in}}}
\newcommand{\est}[1]{\widehat{#1}}
\newcommand{\etab}{\bm{\eta}}
\newcommand{\Exp}[2][]{\mathbb{E}_{#1}\brac{ #2 }} %% Used for Expectation Operator
\newcommand{\expo}[1]{\exp\set{#1}} %% exponential function with dynamic braces
\newcommand{\FC}{{\cal F}}
\renewcommand{\Finv}[2]{F^{-1}_{#1}\paren{#2}} %% inverse CDF function: arg1 for r.v. and arg2 for expression
\newcommand{\finv}[2]{#1^{-1}\paren{#2}} %% inverse function: arg1 is the function; arg2 is the variables
\newcommand{\fnc}[2]{#1\!\paren{#2}} %% function: arg1 is the function; arg2 is the variables
\newcommand{\fncD}[3][D]{#2\!_{#1\!}\!\paren{#3}} %% function: arg1 is the function; arg2 are the variables (used for shape set)
\newcommand{\fncEV}[1]{\bbE\left[ #1 \right]}
									% To property use this function \fncg[subscript]{argument in parentheses}
\newcommand{\g}{\gamma} %% shorthand for lowercase gamma
\newcommand{\gb}{\vecl{g}} %% Bold g
\newcommand{\gbm}{\symbf{\gamma}} %% bold lowercase gamma
\newcommand{\grad}{\nabla} %% Gradient symbol (upside-down Delta)
\newcommand{\hb}{\vecl{h}}
\newcommand{\HC}[1]{\cal H\!_{\textit{#1}}} %% Used to denote the class of homothetic densities
\newcommand{\Ib}{\vecl{I}} %% Bold I for identity matrix
\newcommand{\Ibb}[1]{\mathbb{I}\paren{#1}} %% I for indicator function
\newcommand{\IC}{\cal I}
\newcommand{\iid}{\mathrel{\overset{\text{iid}}{\scalebox{1.5}[1]{$\sim$}}}} %% i.i.d. over distribution symbol
\renewcommand{\iff}{\Leftrightarrow} %shorter if and only if (biconditional)
\newcommand{\ind}{\perp\!\!\!\!\perp} %independence symbol
\newcommand{\inp}[2]{\left\langle #1, #2 \right\rangle} %inner product
\newcommand{\ith}{\text{i}^{\text{th}}}
\newcommand{\jth}{\text{j}^{\text{th}}}
\newcommand{\ka}{\kappa} %% Shortcut for Greek kappa
\newcommand{\kth}{\text{k}^{\text{th}}}
\renewcommand{\L}{\Lambda} %% uppercase lambda
\renewcommand{\l}{\lambda} %% lowercase lambda
\newcommand\latexcode[1]{#1} %% for using accents in Rmarkdown
\newcommand{\lavg}[1]{\overline{#1}} %% place a long bar over the argument
\newcommand{\Lim}[1]{\raisebox{0.5ex}{\scalebox{0.8}{$\displaystyle \lim_{#1}\;$}}} %Place conditions under lim when mixing math mode with text
\newcommand{\limnf}[1]{\displaystyle\lim_{#1\to\nf}} %% Limit to infinity taking argument for variable that goes to infinity; always places text under lim
\newcommand{\lr}{\leftrightarrow} %%two-directional arrow for Markov Chain communicating states
\newcommand{\maxa}[2][]{\max_{#1}\set{#2}} %% Maximum function with braces for argument
\newcommand{\MC}{{\cal M}}
\newcommand{\mcal}[1]{\mathcal{#1}} %% Command for calligraphy \mathcal (like Borel sets)
\newcommand{\mina}[2][]{\min_{#1}\set{#2}} %% Minimum function with braces for argument
\newcommand{\mink}[1]{n_D\paren{#1}} %% Minkowski Functional or Gauge Function
\newcommand{\mmid}{\middle|} %% Dynamically sized vertical bar
\newcommand{\mscr}[1]{\mathscr{#1}} %% Command for script \mathscr (like sigma-algebras)
\newcommand{\mse}[1]{\text{MSE}\paren{#1}}
\newcommand{\mth}{\text{m}^{\text{th}}}
\newcommand{\mub}{\symbf{\mu}} %% Bold mu symbol; mean vector for multivariate normal
\newcommand{\N}{\mathbb{N}} %% Natural numbers
\newcommand{\NC}{{\cal N}}
\newcommand{\ndiv}{\notdivides} %% symbol for "does not divide"
\newcommand{\nf}{\infty} %% Infinity sign
\newcommand{\nlb}{\nolinebreak} %% prevent line breaks and should be used for Equation/Figure/Table references
\newcommand{\norm}[1]{\normalfont{#1}} %% Revert to Normal Font in an italics or bold environment
\newcommand{\nset}{\emptyset} %% Empty Set
\newcommand{\nth}{\text{n}^{\text{th}}}
\newcommand{\nub}{\bm{\nu}} %% Bold latin nu
\renewcommand{\O}{\Omega} %%Capital Omega
\renewcommand{\o}{\omega} %%shorthand for lowercase omega
%% \newcommand{\om}{\omega} %%shorthand for lowercase omega
\newcommand{\OC}{\cal{O}} %% Caligraphy O
\newcommand{\onebb}[1]{\mathbbm{1}\!\paren{#1}} %% Indicator Function
\newcommand{\paren}[1]{\left( #1 \right)} %% Dynamically size parentheses
\newcommand{\Pb}{\vecl{P}} %% Bold P for probability transition matrix (MCMC)
\newcommand{\pb}{\vecl{p}} %% bold lowercase p
\newcommand{\pd}{\partial} %% Shorthand for partial derivative symbol
\newcommand{\pderiv}[1]{\frac{\partial}{\partial #1}} %% partial derivative symbol
\newcommand{\phib}{\bm{\phi}} %% Bold latin phi
\newcommand{\Pois}{\text{Pois}}
\newcommand{\point}[1]{\left( #1 \right)}
\newcommand{\prob}[1]{\mathbb{P}\!\paren{ #1 }} %% Probability as an operator with dynamic sized parentheses
\newcommand{\pset}[1]{\mcal{P}(#1)} %% The Power Set
\newcommand{\psib}{\vecg{\psi}} %% Vector notation for greek letter psi (score functions)
\newcommand{\pth}{p^{\text{th}}}
\newcommand{\Q}{\mathbb{Q}} %% Rational numbers
\newcommand{\Qb}{\vecl{Q}} %% Bold Q for probability transition matrix (MCMC) when creating sets of states
\newcommand{\qb}{\vecl{q}} %% Bold Q for probability transition matrix (MCMC) when creating sets of states
\newcommand{\QC}[1]{\cal Q\!_{\textit{#1}}} %% Used to denote the class of homothetic densities
\newcommand{\R}{\mathbb{R}} %% Real numbers
\newcommand{\Rho}{\mathrm{P}} %% Capital rho
\newcommand{\rhobm}{\symbf{\rho}} %% Bold lowercase rho
\newcommand{\Rplus}{\mathbb{R}^{+}} %% Real numbers
\newcommand{\rb}{\vecl{r}}
\renewcommand\refname{}
\newcommand{\rs}{R^2} %%R-squared
\newcommand{\Rstar}{R^{\star}\!} %%used for star-generalized radius variable
\newcommand{\rstar}{r^{\star}\!} %%used for star-generalized radius variable
\newcommand{\rstari}[1]{r^{\star}_{#1}} %%used for star-generalized radius variable
\renewcommand{\S}{\mathbb{S}} %% Unit sphere
\newcommand{\s}{\sigma} %% lowercase sigma
\newcommand{\salgebra}{\sigma\text{-algebra}}
\newcommand{\salg}{\sigma\text{-algebra}}
\newcommand{\sbf}{\mathbf{s}}
\newcommand{\Sbb}{\mathbb{S}} %% S written for the unit sphere
\newcommand{\SC}{\mcal{S}}
\newcommand{\SD}[1]{\text{SD}\!\left(#1\right)} %% Standard Deviation
\newcommand{\seq}[1]{\left( #1 \right)} %% Sequence operator places dynamic parentheses around #1
\newcommand{\set}[1]{\left\{ #1 \right\}} %Dynamically size braces around an equation
\newcommand{\setm}{\setminus} %%Symmetric difference
\newcommand{\sgn}[1]{\text{sgn}\paren{ #1 }} %sign operator
\newcommand{\Sigmabm}{\bm{{\Sigma}}}
\newcommand{\simp}{\ \Rightarrow \ } %%Shorter arrow for implies
\renewcommand{\ss}{\subset}
\newcommand{\sumnf}[2][1]{\displaystyle\sum_{#2=#1}^{\nf}} %% Limit to infinity taking argument for variable that goes to infinity; always places text under lim
\newcommand{\sumto}[2]{\displaystyle\sum_{#1}^{#2}} %% summation without underscore and carrot
\newcommand{\svar}[1]{\widehat{\text{Var}}\left(#1\right)}
\renewcommand{\T}{\Theta} %% Uppercase Theta
\renewcommand{\t}{\theta} %% Lowercase theta
\newcommand{\tb}{\mathbf{t}} %% Bold-faced lowercase t
\newcommand{\tbf}[1]{\textbf{#1}} %%Bold-faced in text environment
\newcommand{\tbm}{\symbf{\t}} %% Bold-faced lowercase t
\newcommand{\TC}{{\cal T}}
\newcommand{\Tbm}{\symbf{\Theta}} %% Bold-faced capital theta
\newcommand{\tit}[1]{\textit{#1}} %% Italics text
\newcommand{\tPr}{\text{Pr}}
\newcommand{\tr}{^{\intercal}}
\newcommand{\trace}[1]{\text{tr}\paren{#1}} %% trace of matrix
\newcommand{\tth}{\theta} %% Lowercase theta
\newcommand{\ttt}[1]{\texttt{#1}} % Code font in LaTeX
\newcommand{\twonorm}[1]{\left\Vert #1 \right\Vert} % ell-2 (or two-norm) that automatically sizes to the argument
\newcommand{\oneb}{\mathbf{1}}
\newcommand{\Ub}{\mathbf{U}}
\newcommand{\ub}{\mathbf{u}}
\newcommand{\ubrace}[2]{\underbrace{#2}_\text{#1}}
\newcommand{\VaR}[2]{\fnc{\text{VaR}_{#1}}{#2}} %% Value-at-Risk operator
\newcommand{\Var}[2][]{\text{Var}_{#1}\!\left(#2\right)} %% variance operator
\newcommand{\Vb}{\mathbf{V}}
\newcommand{\vb}{\mathbf{v}}
\newcommand{\VC}{{\cal V}}
\newcommand{\vecg}[1]{\bm{#1}}
\newcommand{\vecl}[1]{\mathbf{#1}}
\newcommand{\vphi}{\varphi} %% alternate lowercase phi
\newcommand{\vt}{\vartheta} %% alternate lowercase theta
\def\w{\omega}
\newcommand{\Wb}{\mathbf{W}}
\newcommand{\wb}{\mathbf{w}}
\newcommand{\wbm}{\bm{\w}} %% Bold, lowercase omega
\newcommand{\WC}{\mcal{W}}
\newcommand{\wtilde}[1]{\widetilde{#1}}
\newcommand{\Xb}{\vecl{X}}
\newcommand{\xb}{\mathbf{x}}
\newcommand{\xbm}{\symbf{\xi}} %% Lowercase xi as a vector (bold)
\newcommand{\XC}{\mcal{X}}
\newcommand{\Yb}{\symbf{Y}}
\newcommand{\yb}{\symbf{y}}
\newcommand{\YC}{\mcal{Y}}
\newcommand{\Z}{\mathbb{Z}} %% Denotes the set of integers ..., -2, -1, 0, 1, 2, ...
\newcommand{\Zb}{\mathbf{Z}}
\newcommand{\zb}{\mathbf{z}}
\newcommand{\ZC}{\mcal{Z}}
\newcommand{\zerob}{\mathbf{0}} %%Bold Number 0 for the zero vector

%%%%%%%%%%%%%%%%%%%%%%%%%%%%%%%%%%%%%%%%%%%%%%%%%%%%%%%%%%%%%%%%%%
\DeclarePairedDelimiter{\ellnorm}{\lVert}{\rVert} % ell-q that take size operator in [\big]
\DeclarePairedDelimiter{\floor}{\lfloor}{\rfloor}
\DeclareMathOperator*{\argmin}{argmin}
\DeclareMathOperator*{\argmax}{argmax}
\DeclareMathOperator{\qnorm}{\Phi}
\DeclareMathOperator*{\med}{Med}

%%%%%%%%%%%%%%%%%%%%%%%%%%%%%%%%%%%%%%%%%%%%%%%%%%%%%%%%%%%%%%%%%%
% Glossary Entires
% \usepackage[toc, nonumberlist, nopostdot, style=super, nogroupskip]{glossaries}
% \makeglossaries
% \newacronym{aic}{AIC}{\textit{Akaike's Information Criterion}}
% \newacronym{ar}{A-R}{\textit{Accept-Reject Method}}
% \newacronym{ash}{ASH}{\textit{average shifted histogram}}
% \newacronym{av}{AV}{\tit{asymptotic variance}}
% \newacronym{bic}{BIC}{\textit{Bayesian information criterion}}
% \newacronym{cdf}{CDF}{\textit{cumulative distribution function}}
% \newacronym{cf}{c.f.}{\textit{Characteristic function}}
% \newacronym{clt}{CLT}{\textit{Central Limit Theorem}}
% \newacronym{cvar}{CVaR}{\textit{conditional value-at-risk}}
% \newacronym{cte}{CTE}{\textit{conditional tail expectation}}
% \newacronym{ctp}{CTP}{\textit{conditional tail probability}}
% \newacronym{dcl}{$D$-class}{\textit{directionally dispersed class of distributions}}
% \newacronym{dgf}{dgf}{\textit{density generating function}}
% \newacronym{es}{ES}{\textit{expected shortfall}}
% \newacronym{evt}{EVT}{\textit{extreme value theory}}
% \newacronym{iid}{\tit{i.i.d.}}{\textit{independent and identically distributed}}
% \newacronym{kcv}{$CV_k${\tit{$k$-fold cross-validation}}
% \newacronym{loocv}{LOOCV}{\tit{leave-one-out cross-validation}}
% \newacronym{mad}{MAD}{\tit{median absolute deviation about the median}}
% \newacronym{mcmc}{MCMC}{\tit{Markov chain Monte Carlo}}
% \newacronym{mcte}{MCTE}{\textit{multivariate conditional tail expectation}}
% \newacronym{mctm}{MCTM}{\textit{marginal conditional tail moment}}
% \newacronym{mes}{MES}{\textit{marginal expected shortfall}}
% \newacronym{mh}{MH}{\textit{Metropolis-Hastings}}
% \newacronym{mise}{MISE}{\tit{mean integrated squared error}}
% \newacronym{ml}{ML}{\tit{Maximum Likelihood}}
% \newacronym{mme}{MME}{\textit{marginal mean excess}}
% \newacronym{mse}{MSE}{\tit{mean-squared error}}
% \newacronym{pdf}{PDF}{\textit{probability density function}}
% \newacronym{pmf}{PMF}{\textit{probability mass function}}
% \newacronym{rjmcmc}{RJMCMC}{\textit{reversible-jump Markov chain Monte Carlo}}
% \newacronym{var}{VaR}{\textit{value-at-risk}}
\usepackage{acronym}
\newacro{ae}[a.e.]{\textit{almost everywhere}}
\newacro{aic}[AIC]{\textit{Akaike's Information Criterion}}
\newacro{ar}[A-R]{\textit{Accept-Reject Method}}
\newacro{ash}[ASH]{\textit{average shifted histogram}}
\newacro{av}[AV]{\tit{asymptotic variance}}
\newacro{bic}[BIC]{\textit{Bayesian information criterion}}
\newacro{bp}[BP]{\tit{breakdown point}}
\newacro{cdf}[CDF]{\textit{cumulative distribution function}}
\newacro{cf}[c.f.]{\textit{Characteristic function}}
\newacro{clt}[CLT]{\textit{Central Limit Theorem}}
\newacro{cvar}[CVaR]{\textit{conditional value-at-risk}}
\newacro{cte}[CTE]{\textit{conditional tail expectation}}
\newacro{ctp}[CTP]{\textit{conditional tail probability}}
\newacro{dcl}[$D$-class]{\textit{directionally dispersed class of distributions}}
\newacro{dgf}[dgf]{\textit{density generating function}}
\newacro{es}[ES]{\textit{expected shortfall}}
\newacro{evt}[EVT]{\textit{extreme value theory}}
\newacro{foc}[F.O.C.]{\tit{First Order Conditions}}
\newacro{if}[IF]{\tit{influence function}}
\newacro{gev}[GEV]{\textit{generalized extreme value}}
\newacro{glb}[GLB]{\textit{Greatest Lower Bound}}
\newacro{glr}[GLR]{\tit{Generalized Likelihood Ratio}}
\newacro{iid}[\tit{i.i.d.}]{\textit{independent and identically distributed}}
\newacro{irwls}[IRWLS]{\textit{iteratively reweighted least-squares}}
\newacro{kcv}[$CV_k$]{\tit{$k$-fold cross-validation}}
\newacro{kkt}[KKT]{Karush-Kuhn-Tucker}
\newacro{lln}[LLN]{Law of Large Numbers}
\newacro{lmm}[LMM]{\textit{linear mixed-effect models}}
\newacro{loocv}[LOOCV]{\tit{leave-one-out cross-validation}}
\newacro{ls}[LS]{\tit{least squares}}
\newacro{lsas}[LSAS]{\textit{log-sinh-arcsinh}}
\newacro{lr}[LR]{\textit{Likelihood Ratio}}
\newacro{lub}[LUB]{\textit{Least Upper Bound}}
\newacro{mad}[MAD]{\tit{median absolute deviation about the median}}
\newacro{mcdf}[m.c.d.f.]{\textit{multivariate cumulative distribution function}}
\newacro{mcmc}[MCMC]{\tit{Markov chain Monte Carlo}}
\newacro{mcte}[MCTE]{\textit{multivariate conditional tail expectation}}
\newacro{mctm}[MCTM]{\textit{marginal conditional tail moment}}
\newacro{mes}[MES]{\textit{marginal expected shortfall}}
\newacro{mgf}[MGF]{\textit{moment generating function}}
\newacro{mh}[MH]{\textit{Metropolis-Hastings}}
\newacro{mise}[MISE]{\tit{mean integrated squared error}}
\newacro{ml}[ML]{\tit{Maximum Likelihood}}
\newacro{mle}[MLE]{\tit{maximum likelihood estimate}}
\newacro{mme}[MME]{\textit{marginal mean excess}}
\newacro{mse}[MSE]{\tit{mean-squared error}}
\newacro{ols}[OLS]{ordinary least squares}
\newacro{pc}[PC]{\tit{principal components}}
\newacro{pca}[PCA]{\tit{principal component analysis}}
\newacro{pdf}[PDF]{\textit{probability density function}}
\newacro{pmf}[PMF]{\textit{probability mass function}}
\newacro{rjmcmc}[RJMCMC]{\textit{reversible-jump Markov chain Monte Carlo}}
\newacro{rv}[r.v.]{\textit{random variable}}
\newacro{sas}[SAS]{\textit{sinh-arcsinh}}
\newacro{var}[VaR]{\textit{value-at-risk}}
\newacro{wlln}[WLLN]{\textit{weak law of large numbers}}
\newacro{wlog}[WLOG]{\tit{without loss of generality}}
\newacro{wrt}[w.r.t.]{\tit{with respect to}}