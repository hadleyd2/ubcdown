\PassOptionsToPackage{unicode=true}{hyperref} % options for packages loaded elsewhere
\PassOptionsToPackage{hyphens}{url}
%
\documentclass[]{article}
\usepackage{lmodern}
\usepackage{amssymb,amsmath}
\usepackage{ifxetex,ifluatex}
\usepackage{fixltx2e} % provides \textsubscript
\ifnum 0\ifxetex 1\fi\ifluatex 1\fi=0 % if pdftex
  \usepackage[T1]{fontenc}
  \usepackage[utf8]{inputenc}
  \usepackage{textcomp} % provides euro and other symbols
\else % if luatex or xelatex
  \usepackage{unicode-math}
  \defaultfontfeatures{Ligatures=TeX,Scale=MatchLowercase}
\fi
% use upquote if available, for straight quotes in verbatim environments
\IfFileExists{upquote.sty}{\usepackage{upquote}}{}
% use microtype if available
\IfFileExists{microtype.sty}{%
\usepackage[]{microtype}
\UseMicrotypeSet[protrusion]{basicmath} % disable protrusion for tt fonts
}{}
\IfFileExists{parskip.sty}{%
\usepackage{parskip}
}{% else
\setlength{\parindent}{0pt}
\setlength{\parskip}{6pt plus 2pt minus 1pt}
}
\usepackage{hyperref}
\hypersetup{
            pdfborder={0 0 0},
            breaklinks=true}
\urlstyle{same}  % don't use monospace font for urls
\usepackage[margin=1in]{geometry}
\usepackage{longtable,booktabs}
% Fix footnotes in tables (requires footnote package)
\IfFileExists{footnote.sty}{\usepackage{footnote}\makesavenoteenv{longtable}}{}
\usepackage{graphicx,grffile}
\makeatletter
\def\maxwidth{\ifdim\Gin@nat@width>\linewidth\linewidth\else\Gin@nat@width\fi}
\def\maxheight{\ifdim\Gin@nat@height>\textheight\textheight\else\Gin@nat@height\fi}
\makeatother
% Scale images if necessary, so that they will not overflow the page
% margins by default, and it is still possible to overwrite the defaults
% using explicit options in \includegraphics[width, height, ...]{}
\setkeys{Gin}{width=\maxwidth,height=\maxheight,keepaspectratio}
\setlength{\emergencystretch}{3em}  % prevent overfull lines
\providecommand{\tightlist}{%
  \setlength{\itemsep}{0pt}\setlength{\parskip}{0pt}}
\setcounter{secnumdepth}{5}
% Redefines (sub)paragraphs to behave more like sections
\ifx\paragraph\undefined\else
\let\oldparagraph\paragraph
\renewcommand{\paragraph}[1]{\oldparagraph{#1}\mbox{}}
\fi
\ifx\subparagraph\undefined\else
\let\oldsubparagraph\subparagraph
\renewcommand{\subparagraph}[1]{\oldsubparagraph{#1}\mbox{}}
\fi

% set default figure placement to htbp
\makeatletter
\def\fps@figure{htbp}
\makeatother

%%%%%%%%%%%%%%%%%%%%%%%%%%%%%%%%%%%%%%%%%%%%%%%%%%%%%%%%%%%%%%%%%%
\usepackage{amsmath}
\usepackage{xpatch}
\makeatletter
\makeatother
\usepackage{amssymb, amsfonts, latexsym, fullpage}
% \usepackage{multirow, hhline} %% To build tables with different sized rows/double dividing lines
%%%% Packages taken from the ubcdown main_script.Rmd template
\usepackage{ragged2e}
\usepackage{setspace}
\usepackage{tocloft}
\usepackage{float} 
\floatplacement{figure}{H} 
\usepackage{wrapfig}
\usepackage{pdfpages}
\usepackage{mathtools}
\usepackage{bm} %For bold lowercase Greek letters
\usepackage{natbib} %For references
\usepackage{xcolor} %Allows us to set color for comments
\usepackage{caption} %using this package to control the font size in the table and figure captions
\captionsetup{font=small} % \captionsetup{font=footnotesize}
\usepackage[linesnumbered,ruled]{algorithm2e} %To write algorithms in a neat format
\usepackage{mathrsfs} %Let's us use \mathscr{} to produce script letters
\usepackage{graphicx} %% Allows specification of a directory
%\graphicspath{ {"/Users/danielhadley/Documents/SUMMER 2019/Research/Hadley/R Files/Images/"} }


% \usepackage{xypic} %% For drawing vines
% \usepackage{mathabx} %% For the \notdivides symbol
% \usepackage{xinttools} %% For making dynamic tables in latex
% \usepackage{amsthm} %% Trying to get theorems to start after a line break.
% % \usepackage{showlabels} %% Allows us to see the labels given to equation, figures, etc. for easy reference


%\linespread{1.6}
\newtheorem{theorem}{Theorem}
% \newtheorem*{theorem*}{Theorem}
\newtheorem{lemma}[theorem]{Lemma}
% \newtheorem*{lemma*}{Lemma}
\newtheorem{proposition}[theorem]{Proposition}
% \newtheorem*{proposition*}{Proposition}
\newtheorem{corollary}[theorem]{Corollary}
% \newtheorem*{corollary*}{Corollary}
\newtheorem{defn}{Definition}
\newtheorem{definition}{Definition}
% \newtheorem*{definition*}{Definition}
% \newtheorem*{example*}{Example}
\newtheorem{example}{Example}
\newtheorem{exercise}{Exercise}
\newtheorem{assumption}{Assumption}[section]

%%%%%%%%%%%%%%%%%%%%%%%%%%%%%%%%%%%%%%%%%%%%%%%%%%%%%%%%%%%%%%%%%%
%% Extra environment provides a shaded box
% \usepackage{xcolor}
% \usepackage{comment}
% \definecolor{shadecolor}{gray}{0.9}
% \specialcomment{extra}{\begin{shaded}}{\end{shaded}}


%%%%%%%%%%%%%%%%%%%%%%%%%%%%%%%%%%%%%%%%%%%%%%%%%%%%%%%%%%%%%%%%%%
%% New environment for small font matrix
% \newenvironment{smat}
%   {\left[\begin{smallmatrix}}
%   {\end{smallmatrix}\right]}
  
\newcommand{\smat}[1]{\brac{\begin{smallmatrix} #1 \end{smallmatrix}}}

%% New environment for box matrix
\newcommand{\bmat}[1]{\begin{bmatrix} #1 \end{bmatrix}}

%% New environment for parentheses matrix
\newcommand{\pmat}[1]{\begin{pmatrix} #1 \end{pmatrix}}

%% New environment to number one line of an align* environment
\newcommand\numberthis{\addtocounter{equation}{1}\tag{\theequation}}

%%%%%%%%%%%%%%%%%%%%%%%%%%%%%%%%%%%%%%%%%%%%%%%%%%%%%%%%%%%%%%%%%%
%% Equation Environment so I don't have to write out equation* all the time
\newcommand{\eeq}[2][]{\begin{equation} \label{Eq:#1} #2 \end{equation}}

%%%%%%%%%%%%%%%%%%%%%%%%%%%%%%%%%%%%%%%%%%%%%%%%%%%%%%%%%%%%%%%%%%
%% Equation Environment so I don't have to write out equation* all the time
\newcommand{\eqq}[2][]{\begin{equation*} #2 \end{equation*}}

%%%%%%%%%%%%%%%%%%%%%%%%%%%%%%%%%%%%%%%%%%%%%%%%%%%%%%%%%%%%%%%%%%
%% Number equation by their section
\numberwithin{equation}{section}

%%%%%%%%%%%%%%%%%%%%%%%%%%%%%%%%%%%%%%%%%%%%%%%%%%%%%%%%%%%%%%%%%%
%% Environment for importing R files as code and formatting options
\usepackage{listings} %For adding line numbers to
\lstset
{ %Formatting for code in appendix
    language=R,
    basicstyle=\footnotesize,
    numbers=left,
    stepnumber=1,
    showstringspaces=false,
    tabsize=1,
    breaklines=true,
    breakatwhitespace=false,
}

%%%%%%%%%%%%%%%%%%%%%%%%%%%%%%%%%%%%%%%%%%%%%%%%%%%%%%%%%%%%%%%%%%
%% Matrix environment shortcut
\def\bma{\begin{pmatrix}}
\def\ema{\end{pmatrix}}

%% Allows specification of a directory
\usepackage{graphicx}
%\graphicspath{ {"/Users/danielhadley/Documents/SUMMER 2019/Research/Hou-Thesis/"} }

\setcounter{tocdepth}{5}
\setcounter{secnumdepth}{5}

%%%%%%%%%%%%%%%%%%%%%%%%%%%%%%%%%%%%%%%%%%%%%%%%%%%%%%%%%%%%%%%%%%
%% Create shortcuts for the set of real numbers, natural numbers, rational numbers, and infinity
\renewcommand{\a}{\alpha}
\newcommand{\ab}{\vecl{a}}
\newcommand{\abm}{\bm{\alpha}}
\newcommand{\abs}[1]{\left\vert #1 \right\vert} % Absolute Value that take size operator in [\big]
\newcommand{\AC}{\mcal{A}}
\newcommand{\alpb}{\vecg{\alpha}}
\newcommand{\AS}{\mscr{A}} % Script A for sets of states in MCMC
\newcommand{\avg}[1]{\overline{#1}}
\renewcommand{\b}{\beta}
\newcommand{\Bb}{\vecl{B}}
\newcommand{\bb}{\vecl{b}}
\newcommand{\bbI}{\mathbbm{I}}
\newcommand{\bbone}{\mathbbm{1}}
\newcommand{\bbP}[1]{\mathbb{P}\paren{ #1 }} %% Probability as an operator
\newcommand{\BC}{\mcal{B}}
\newcommand{\bdry}[1]{\pd\!#1} %% Function to denote a boundary
\newcommand{\betab}{\vecg{\beta}}
\newcommand{\bi}{\Leftrightarrow} %%a biconditional arrow
\newcommand{\bias}[1]{\text{Bias}\paren{#1}}
\newcommand{\brac}[1]{\left[ #1 \right]} %% Brackets that dynamically sizene
\renewcommand{\C}{\mathbb{C}} %% Complex numbers
\newcommand{\cas}{\stackrel{as}{\to}} %% converges almost surely
\newcommand{\cb}{\mathbf{c}}
\newcommand{\CC}{\mcal{C}}
\newcommand{\ceil}[1]{\left\lceil #1 \right\rceil} %% Ceiling function with dynamic sizing
\newcommand{\cid}{\stackrel{d}{\to}} %% convergence in distribution symbol
\newcommand{\cip}{\stackrel{p}{\to}} %% convergence in probability symbol
\newcommand{\cond}{\ \middle| \ }
\newcommand{\Corr}[1]{\text{Corr}\!\left(#1\right)} %% write Corr as text in math mode for correlation operator
\newcommand{\Cov}[1]{\text{Cov}\!\left(#1\right)} %% write Cov as text in math mode for covariance operator
\newcommand{\D}{\Delta} %% Uppercase delta
\newcommand{\de}{\delta} %% Lowercase delta
\newcommand{\deriv}[1]{\frac{d}{d #1}} %% univariate derivative symbol
\newcommand{\Db}{\mathbf{D}} %% bold capital latin letter D
\newcommand{\DC}{{\cal D}}
\newcommand{\db}{\mathbf{d}} %% bold lowercase latin letter D
\newcommand{\diag}[1]{\text{diag}\paren{#1}}
\newcommand{\dist}[3][]{\fnc{\text{#2}_{#1}}{#3}}
\newcommand{\dom}[1]{\mathrm{dom}\paren{#1}} %% domain of a function
\newcommand{\dotb}{\bullet} %% slightly bigger cdot symbol for dot product
\newcommand{\Eb}{\vecl{E}}
\newcommand{\eb}{\vecl{e}}
\newcommand{\EC}{{\cal E}}
\newcommand{\eid}{\stackrel{d}{=}} %% equal in distribution
\newcommand{\ep}{\epsilon}
\newcommand{\epb}{\vecg{\epsilon}}
\newcommand{\epp}{\varepsilon}
\newcommand{\eppbm}{\bm{\varepsilon}} %% bold lowercase epsilon variation
\newcommand{\eqstar}{\stackrel{*}{=}} %% equal sign with a star over it for a footnote
\newcommand{\errtr}{\text{Err}_{\text{tr}}}
\newcommand{\errin}{\text{Err}_{\text{in}}}
\newcommand{\est}[1]{\widehat{#1}}
\newcommand{\etab}{\bm{\eta}}
\newcommand{\Exp}[2][]{\mathbb{E}_{#1}\brac{ #2 }} %% Used for Expectation Operator
\newcommand{\expo}[1]{\exp\set{#1}} %% exponential function with dynamic braces
\newcommand{\FC}{{\cal F}}
\renewcommand{\Finv}[2]{F^{-1}_{#1}\paren{#2}} %% inverse CDF function: arg1 for r.v. and arg2 for expression
\newcommand{\finv}[2]{#1^{-1}\paren{#2}} %% inverse function: arg1 is the function; arg2 is the variables
\newcommand{\fnc}[2]{#1\!\paren{#2}} %% function: arg1 is the function; arg2 is the variables
\newcommand{\fncD}[3][D]{#2\!_{#1\!}\!\paren{#3}} %% function: arg1 is the function; arg2 are the variables (used for shape set)
\newcommand{\fncDtest}[3][D]{#2\!_{#1\!}\!\paren{#3}} %% function: arg1 is the function; arg2 are the variables (used for shape set)
\newcommand{\fncEV}[1]{\bbE\left[ #1 \right]}
\newcommand{\fncf}[2][]{f_{#1}\left( #2 \right)} %For function f() with dynamically sized parentheses and optional argument for subscript
									% To property use this function \fncf[subscript]{argument in parentheses}
\newcommand{\fncg}[2][]{g_{#1}\left( #2 \right)} %For function g() with dynamically sized parentheses and optional argument for subscript
									% To property use this function \fncg[subscript]{argument in parentheses}
\newcommand{\g}{\gamma} %% shorthand for lowercase gamma
\newcommand{\gb}{\vecl{g}} %% Bold g
\newcommand{\gbm}{\bm{{\gamma}}} %% bold lowercase gamma
\newcommand{\grad}{\nabla} %% Gradient symbol (upside-down Delta)
\newcommand{\hb}{\vecl{h}}
\newcommand{\HC}[1]{\cal{H}\!_{\mathit{#1}}} %% Used to denote the class of homothetic densities
\newcommand{\Ib}{\vecl{I}} %% Bold I for identity matrix
\newcommand{\Ibb}[1]{\mathbb{I}\paren{#1}} %% I for indicator function
\newcommand{\IC}{\cal I}
\newcommand{\iid}{\mathrel{\overset{\text{iid}}{\scalebox{1.5}[1]{$\sim$}}}} %% i.i.d. over distribution symbol
\renewcommand{\iff}{\Leftrightarrow} %shorter if and only if (biconditional)
\newcommand{\ind}{\perp\!\!\!\!\perp} %independence symbol
\newcommand{\inp}[2]{\left\langle #1, #2 \right\rangle} %inner product
\newcommand{\ith}{\text{i}^{\text{th}}}
\newcommand{\jth}{\text{j}^{\text{th}}}
\newcommand{\ka}{\kappa} %% Shortcut for Greek kappa
\newcommand{\kth}{\text{k}^{\text{th}}}
\renewcommand{\L}{\Lambda} %% uppercase lambda
\renewcommand{\l}{\lambda} %% lowercase lambda
\newcommand\latexcode[1]{#1} %% for using accents in Rmarkdown
\newcommand{\lavg}[1]{\overline{#1}} %% place a long bar over the argument
\newcommand{\Lim}[1]{\raisebox{0.5ex}{\scalebox{0.8}{$\displaystyle \lim_{#1}\;$}}} %Place conditions under lim when mixing math mode with text
\newcommand{\limnf}[1]{\displaystyle\lim_{#1\to\nf}} %% Limit to infinity taking argument for variable that goes to infinity; always places text under lim
\newcommand{\lr}{\leftrightarrow} %%two-directional arrow for Markov Chain communicating states
\newcommand{\maxa}[2][]{\max_{#1}\set{#2}} %% Maximum function with braces for argument
\newcommand{\MC}{{\cal M}}
\newcommand{\mcal}[1]{\mathcal{#1}} %% Command for calligraphy \mathcal (like Borel sets)
\newcommand{\mina}[2][]{\min_{#1}\set{#2}} %% Minimum function with braces for argument
\newcommand{\mink}[1]{n_D\paren{#1}} %% Minkowski Functional or Gauge Function
\newcommand{\mmid}{\middle|} %% Dynamically sized vertical bar
\newcommand{\mscr}[1]{\mathscr{#1}} %% Command for script \mathscr (like sigma-algebras)
\newcommand{\mse}[1]{\text{MSE}\paren{#1}}
\newcommand{\mth}{\text{m}^{\text{th}}}
\newcommand{\mub}{\bm{{\mu}}} %% Bold mu symbol; mean vector for multivariate normal
\newcommand{\N}{\mathbb{N}} %% Natural numbers
\newcommand{\NC}{{\cal N}}
\newcommand{\ndiv}{\notdivides} %% symbol for "does not divide"
\newcommand{\nf}{\infty} %% Infinity sign
\newcommand{\nlb}{\nolinebreak} %% prevent line breaks and should be used for Equation/Figure/Table references
\newcommand{\norm}[1]{\normalfont{#1}} %% Revert to Normal Font in an italics or bold environment
\newcommand{\nset}{\emptyset} %% Empty Set
\newcommand{\nth}{\text{n}^{\text{th}}}
\newcommand{\nub}{\bm{\nu}} %% Bold latin nu
\renewcommand{\O}{\Omega} %%Capital Omega
\renewcommand{\o}{\omega} %%shorthand for lowercase omega
%% \newcommand{\om}{\omega} %%shorthand for lowercase omega
\newcommand{\OC}{\cal{O}} %% Caligraphy O
\newcommand{\onebb}[1]{\mathbbm{1}\!\paren{#1}} %% Indicator Function
\newcommand{\paren}[1]{\left( #1 \right)} %% Dynamically size parentheses
\newcommand{\Pb}{\vecl{P}} %% Bold P for probability transition matrix (MCMC)
\newcommand{\pb}{\vecl{p}} %% bold lowercase p
\newcommand{\pd}{\partial} %% Shorthand for partial derivative symbol
\newcommand{\pderiv}[1]{\frac{\partial}{\partial #1}} %% partial derivative symbol
\newcommand{\phib}{\bm{\phi}} %% Bold latin phi
\newcommand{\Pois}{\text{Pois}}
\newcommand{\point}[1]{\left( #1 \right)}
\newcommand{\prob}[1]{\bbP{ #1}} %% Probability as an operator with dynamic sized parentheses
\newcommand{\pset}[1]{\mcal{P}(#1)} %% The Power Set
\newcommand{\psib}{\vecg{\psi}} %% Vector notation for greek letter psi (score functions)
\newcommand{\pth}{p^{\text{th}}}
\newcommand{\Q}{\mathbb{Q}} %% Rational numbers
\newcommand{\Qb}{\vecl{Q}} %% Bold Q for probability transition matrix (MCMC) when creating sets of states
\newcommand{\qb}{\vecl{q}} %% Bold Q for probability transition matrix (MCMC) when creating sets of states
\newcommand{\QC}{\mcal{Q}}
\newcommand{\R}{\mathbb{R}} %% Real numbers
\newcommand{\Rho}{\mathrm{P}} %% Capital rho
\newcommand{\rhobm}{\vecg{\rho}} %% Bold lowercase rho
\newcommand{\Rplus}{\mathbb{R}^{+}} %% Real numbers
\newcommand{\rb}{\vecl{r}}
\newcommand{\rs}{R^2} %%R-squared
\newcommand{\Rstar}{R^{\star}\!} %%used for star-generalized radius variable
\newcommand{\rstar}{r^{\star}\!} %%used for star-generalized radius variable
\newcommand{\rstari}[1]{r^{\star}_{#1}} %%used for star-generalized radius variable
\renewcommand{\S}{\mathbb{S}} %% Unit sphere
\newcommand{\s}{\sigma} %% lowercase sigma
\newcommand{\salgebra}{\sigma\text{-algebra}}
\newcommand{\salg}{\sigma\text{-algebra}}
\newcommand{\sbf}{\mathbf{s}}
\newcommand{\Sbb}{\mathbb{S}} %% S written for the unit sphere
\newcommand{\SC}{\mcal{S}}
\newcommand{\SD}[1]{\text{SD}\!\left(#1\right)} %% Standard Deviation
\newcommand{\seq}[1]{\left( #1 \right)} %% Sequence operator places dynamic parentheses around #1
\newcommand{\set}[1]{\left\{ #1 \right\}} %Dynamically size braces around an equation
\newcommand{\setm}{\setminus} %%Symmetric difference
\newcommand{\sgn}[1]{\text{sgn}\paren{ #1 }} %sign operator
\newcommand{\Sigmabm}{\bm{\Sigma}}
\newcommand{\simp}{\ \Rightarrow \ } %%Shorter arrow for implies
\renewcommand{\ss}{\subset}
\newcommand{\sumnf}[2][1]{\displaystyle\sum_{#2=#1}^{\nf}} %% Limit to infinity taking argument for variable that goes to infinity; always places text under lim
\newcommand{\sumto}[2]{\displaystyle\sum_{#1}^{#2}} %% summation without underscore and carrot
\newcommand{\svar}[1]{\widehat{\text{Var}}\left(#1\right)}
\renewcommand{\T}{\Theta} %% Uppercase Theta
\renewcommand{\t}{\theta} %% Lowercase theta
\newcommand{\tb}{\mathbf{t}} %% Bold-faced lowercase t
\newcommand{\tbf}[1]{\textbf{#1}} %%Bold-faced in text environment
\newcommand{\tbm}{\bm{\t}} %% Bold-faced lowercase t
\newcommand{\TC}{{\cal T}}
\newcommand{\Tbm}{\bm{\Theta}} %% Bold-faced capital theta
\newcommand{\thb}{\bm{\theta}} %% Bold-faced lowercase theta
\newcommand{\tit}[1]{\textit{#1}} %% Italics text
\newcommand{\tPr}{\text{Pr}}
\newcommand{\tr}{^{\intercal}}
\newcommand{\trace}[1]{\text{tr}\paren{#1}} %% trace of matrix
\newcommand{\tth}{\theta} %% Lowercase theta
\newcommand{\ttt}[1]{\texttt{#1}} % Code font in LaTeX
\newcommand{\twonorm}[1]{\left\Vert #1 \right\Vert} % ell-2 (or two-norm) that automatically sizes to the argument
\newcommand{\oneb}{\mathbf{1}}
\newcommand{\Ub}{\mathbf{U}}
\newcommand{\ub}{\mathbf{u}}
\newcommand{\ubrace}[2]{\underbrace{#2}_\text{#1}}
\newcommand{\VaR}[2]{\fnc{\text{VaR}_{#1}}{#2}} %% Value-at-Risk operator
\newcommand{\Var}[2][]{\text{Var}_{#1}\!\left(#2\right)} %% variance operator
\newcommand{\Vb}{\mathbf{V}}
\newcommand{\vb}{\mathbf{v}}
\newcommand{\VC}{{\cal V}}
\newcommand{\vecg}[1]{\bm{#1}}
\newcommand{\vecl}[1]{\mathbf{#1}}
\newcommand{\vphi}{\varphi} %% alternate lowercase phi
\newcommand{\vt}{\vartheta} %% alternate lowercase theta
\def\w{\omega}
\newcommand{\Wb}{\mathbf{W}}
\newcommand{\wb}{\mathbf{w}}
\newcommand{\wbm}{\bm{\w}} %% Bold, lowercase omega
\newcommand{\WC}{\mcal{W}}
\newcommand{\wtilde}[1]{\widetilde{#1}}
\newcommand{\Xb}{\vecl{X}}
\newcommand{\xb}{\mathbf{x}}
\newcommand{\xbm}{\vecg{\xi}} %% Lowercase xi as a vector (bold)
\newcommand{\XC}{\mcal{X}}
\newcommand{\Yb}{\vecl{Y}}
\newcommand{\yb}{\vecl{y}}
\newcommand{\YC}{\mcal{Y}}
\newcommand{\Z}{\mathbb{Z}} %% Denotes the set of integers ..., -2, -1, 0, 1, 2, ...
\newcommand{\Zb}{\mathbf{Z}}
\newcommand{\zb}{\mathbf{z}}
\newcommand{\ZC}{\mcal{Z}}
\newcommand{\zerob}{\mathbf{0}} %%Bold Number 0 for the zero vector

%%%%%%%%%%%%%%%%%%%%%%%%%%%%%%%%%%%%%%%%%%%%%%%%%%%%%%%%%%%%%%%%%%
\DeclarePairedDelimiter{\ellnorm}{\lVert}{\rVert} % ell-q that take size operator in [\big]
\DeclarePairedDelimiter{\floor}{\lfloor}{\rfloor}
\DeclareMathOperator*{\argmin}{argmin}
\DeclareMathOperator*{\argmax}{argmax}
\DeclareMathOperator{\qnorm}{\Phi}
\DeclareMathOperator*{\med}{Med}

%%%%%%%%%%%%%%%%%%%%%%%%%%%%%%%%%%%%%%%%%%%%%%%%%%%%%%%%%%%%%%%%%%
% Glossary Entires
\usepackage{acronym}
\newacro{ae}[a.e.]{\textit{almost everywhere}}
\newacro{aic}[AIC]{\textit{Akaike's Information Criterion}}
\newacro{ar}[A-R]{\textit{Accept-Reject Method}}
\newacro{ash}[ASH]{\textit{average shifted histogram}}
\newacro{av}[AV]{\tit{asymptotic variance}}
\newacro{bic}[BIC]{\textit{Bayesian information criterion}}
\newacro{bp}[BP]{\tit{breakdown point}}
\newacro{cdf}[CDF]{\textit{cumulative distribution function}}
\newacro{cf}[c.f.]{\textit{Characteristic function}}
\newacro{clt}[CLT]{\textit{Central Limit Theorem}}
\newacro{cvar}[CVaR]{\textit{conditional value-at-risk}}
\newacro{cte}[CTE]{\textit{conditional tail expectation}}
\newacro{ctp}[CTP]{\textit{conditional tail probability}}
\newacro{dcl}[$D$-class]{\textit{directionally dispersed class of distributions}}
\newacro{dgf}[dgf]{\textit{density generating function}}
\newacro{es}[ES]{\textit{expected shortfall}}
\newacro{evt}[EVT]{\textit{extreme value theory}}
\newacro{foc}[F.O.C.]{\tit{First Order Conditions}}
\newacro{if}[IF]{\tit{influence function}}
\newacro{gev}[GEV]{\textit{generalized extreme value}}
\newacro{glb}[GLB]{\textit{Greatest Lower Bound}}
\newacro{glr}[GLR]{\tit{Generalized Likelihood Ratio}}
\newacro{iid}[\tit{i.i.d.}]{\textit{independent and identically distributed}}
\newacro{irwls}[IRWLS]{\textit{iteratively reweighted least-squares}}
\newacro{kcv}[$CV_k$]{\tit{$k$-fold cross-validation}}
\newacro{kkt}[KKT]{Karush-Kuhn-Tucker}
\newacro{lln}[LLN]{Law of Large Numbers}
\newacro{lmm}[LMM]{\textit{linear mixed-effect models}}
\newacro{loocv}[LOOCV]{\tit{leave-one-out cross-validation}}
\newacro{ls}[LS]{\tit{least squares}}
\newacro{lsas}[LSAS]{\textit{log-sinh-arcsinh}}
\newacro{lr}[LR]{\textit{Likelihood Ratio}}
\newacro{lub}[LUB]{\textit{Least Upper Bound}}
\newacro{mad}[MAD]{\tit{median absolute deviation about the median}}
\newacro{mcdf}[m.c.d.f.]{\textit{multivariate cumulative distribution function}}
\newacro{mcmc}[MCMC]{\tit{Markov chain Monte Carlo}}
\newacro{mcte}[MCTE]{\textit{multivariate conditional tail expectation}}
\newacro{mctm}[MCTM]{\textit{marginal conditional tail moment}}
\newacro{mes}[MES]{\textit{marginal expected shortfall}}
\newacro{mgf}[MGF]{\textit{moment generating function}}
\newacro{mh}[MH]{\textit{Metropolis-Hastings}}
\newacro{mise}[MISE]{\tit{mean integrated squared error}}
\newacro{ml}[ML]{\tit{Maximum Likelihood}}
\newacro{mle}[MLE]{\tit{maximum likelihood estimate}}
\newacro{mme}[MME]{\textit{marginal mean excess}}
\newacro{mse}[MSE]{\tit{mean-squared error}}
\newacro{ols}[OLS]{ordinary least squares}
\newacro{pc}[PC]{\tit{principal components}}
\newacro{pca}[PCA]{\tit{principal component analysis}}
\newacro{pdf}[PDF]{\textit{probability density function}}
\newacro{pmf}[PMF]{\textit{probability mass function}}
\newacro{rjmcmc}[RJMCMC]{\textit{reversible-jump Markov chain Monte Carlo}}
\newacro{rv}[r.v.]{\textit{random variable}}
\newacro{sas}[SAS]{\textit{sinh-arcsinh}}
\newacro{var}[VaR]{\textit{value-at-risk}}
\newacro{wlln}[WLLN]{\textit{weak law of large numbers}}
\newacro{wlog}[WLOG]{\tit{without loss of generality}}
\newacro{wrt}[w.r.t.]{\tit{with respect to}}
\usepackage{booktabs}
\usepackage{longtable}
\usepackage{array}
\usepackage{multirow}
\usepackage{wrapfig}
\usepackage{float}
\usepackage{colortbl}
\usepackage{pdflscape}
\usepackage{tabu}
\usepackage{threeparttable}
\usepackage{threeparttablex}
\usepackage[normalem]{ulem}
\usepackage{makecell}
\usepackage{xcolor}

\author{}
\date{\vspace{-2.5em}}

\begin{document}

\thispagestyle{empty}
\begin{singlespace} 
\begin{center}

\Huge

\textbf{HERE GOES YOUR TITLE}

\Large

\vspace{8mm}

by

\vspace{8mm}

\huge

Here Goes Your Name

\Large

\vspace{8mm}

Undergraduate degree

Master degree

\vspace{10mm}

A DISSERTATION SUBMITTED IN PARTIAL FULFILLMENT OF THE REQUIREMENTS FOR THE DEGREE OF

\vspace{5mm}

HERE GOES YOUR DEGREEE

\vspace{5mm}
in

\vspace{5mm}

THE FACULTY OF GRADUATE AND POSTDOCTORAL STUDIES

\vspace{5mm}

(Program)

\vspace{5mm}

THE UNIVERSITY OF BRITISH COLUMBIA

\vspace{5mm}

(Vancouver)

\vspace{6mm}

Month YYYY

© Here Goes Your Name, YYYY

\end{center}
\end{singlespace}
\normalsize
\clearpage

\pagenumbering{roman}
\setcounter{page}{2}

\doublespacing

The following individuals certify that they have read, and recommend to the Faculty of Graduate
and Postdoctoral Studies for acceptance, the dissertation entitled:

\vspace{5mm}

\noindent\underline{\makebox[6in][l]{Insert your dissertation title here}}\\
\vspace{2mm}

submitted by \underline{Insert Your Name Here} in partial fulfillment of the requirements for

the degree of \noindent\underline{\makebox[5in][l]{Name of your Degree}}

in \noindent\underline{\makebox[5.8in][l]{Insert Program}}

\vspace{5mm}

\textbf{Examining Committee:}

\noindent\underline{\makebox[6in][l]{Name of your Supervisor}}\\
Research Supervisor

\noindent\underline{\makebox[6in][l]{Name of your Committee Member}}\\
Supervisory Committee Member

\noindent\underline{\makebox[6in][l]{Name of University Examiner}}\\
University Examiner

\noindent\underline{\makebox[6in][l]{Name of University Examiner}}\\
University Examiner

\noindent\underline{\makebox[6in][l]{Name of External Examiner}}\\
External Examiner

\clearpage

\setlength{\parindent}{4em} 
\linespread{1}
\doublespacing

\section*{Abstract}
\addcontentsline{toc}{section}{Abstract}

Lorem ipsum dolor sit amet, consectetur adipiscing elit. Nulla eleifend odio odio, et consectetur nunc commodo vitae. Donec eu finibus lacus. Integer at porta nulla. Fusce dignissim gravida lacinia. Nulla ut odio in ante imperdiet placerat. Maecenas vel imperdiet lorem. Proin ultricies, ex at bibendum fringilla, neque libero malesuada sem, in tincidunt leo mi vel mi. Phasellus scelerisque sagittis urna sed sodales. Curabitur iaculis justo velit, ultricies euismod justo convallis non. Nunc id purus et ante iaculis dignissim vel non erat. Donec dignissim urna ut lobortis congue. Nulla tempus ut ipsum non ornare. Donec ac purus nec sem cursus eleifend. Suspendisse potenti. Vestibulum ac vehicula ante.

Donec pulvinar a est ut vulputate. In id euismod nisi. Curabitur pulvinar bibendum convallis. Phasellus luctus posuere quam, a dapibus massa tristique in. Sed fermentum fermentum velit non sodales. Donec eget efficitur turpis. Sed ligula diam, ultrices porta vestibulum sed, varius maximus erat. Sed sagittis ut nisi ac tincidunt. Aliquam augue elit, egestas eu commodo ut, eleifend et nulla. Phasellus mollis lacus ac tortor fringilla, eget viverra enim sagittis. Morbi facilisis porttitor dui, eget elementum tellus malesuada ac. Integer sed accumsan lorem. Cras placerat ultrices libero at efficitur. Orci varius natoque penatibus et magnis dis parturient montes, nascetur ridiculus mus.

Nullam magna elit, sodales eu massa a, rutrum gravida orci. Mauris cursus odio turpis, et dignissim neque rutrum sit amet. Mauris vel nibh enim. Vestibulum tempus et lectus quis tempor. Proin dictum dui luctus felis sodales porttitor. Maecenas viverra vestibulum tristique. Nulla facilisi. Quisque efficitur lacus diam. Phasellus eget ligula quis mauris fermentum feugiat. Duis vitae lectus porttitor, vestibulum velit a, tincidunt nulla. Donec pharetra erat et urna sollicitudin, ac iaculis quam pulvinar. Donec pulvinar justo maximus lacinia sodales. Etiam sit amet diam vel ligula tempor congue nec in libero. Donec arcu risus, posuere.

\clearpage

\section*{Lay Summary}
\addcontentsline{toc}{section}{Lay Summary}

Lorem ipsum dolor sit amet, consectetur adipiscing elit. Sed lorem massa, suscipit vel dapibus a, iaculis ut sapien. Duis eu tellus vitae ligula semper commodo. Morbi felis tortor, ullamcorper vitae venenatis sit amet, commodo eu tortor. Proin ultrices ipsum ipsum, sed viverra arcu maximus eget. Aliquam nec nisl rutrum quam tempus iaculis nec sed dui. Nulla vitae imperdiet neque. Morbi eget eleifend lectus. Etiam semper id purus et scelerisque. Suspendisse lectus libero, iaculis et arcu vel, mattis mollis odio.

Phasellus quis metus erat. Nunc erat mi, posuere eleifend tortor quis, consequat dignissim metus. Pellentesque eget ante massa. Integer ultrices pulvinar varius. Sed dapibus sed arcu sit amet cursus. In hac habitasse platea dictumst. Aliquam vehicula est sed velit dapibus commodo. Donec placerat eu metus vel efficitur. Pellentesque commodo, nisi at elementum sagittis, erat magna feugiat elit, sit amet facilisis est diam eget tellus. Ut tellus velit, fringilla non consequat ut.

\clearpage

\section*{Preface}
\addcontentsline{toc}{section}{Preface}

Lorem ipsum dolor sit amet, consectetur adipiscing elit. Cras id nulla malesuada, auctor arcu at, tempor nibh. Etiam ultricies dictum nisi eu vulputate. Suspendisse tempor aliquam efficitur. Curabitur eget orci ac purus hendrerit laoreet. Proin porta arcu nisl, a fermentum tellus suscipit et. Nulla mattis orci ac tortor facilisis venenatis. Phasellus a velit tristique, tincidunt purus id, aliquet tellus. Orci varius natoque penatibus et magnis dis parturient montes, nascetur ridiculus mus. Sed et turpis sed odio faucibus aliquam nec eget est.

Phasellus ligula lacus, euismod posuere pretium sit amet, ornare condimentum nisl. Integer eu aliquet lectus, sit amet sagittis turpis. Etiam tempus bibendum nibh, eu accumsan sapien. Suspendisse tincidunt sit amet nisi vel consectetur. Donec vitae odio elit. Quisque dignissim risus feugiat nunc tempor hendrerit. Fusce consequat lectus eget felis tempus, et lobortis ante ornare. Etiam euismod quam eu scelerisque mattis. Praesent nisi lectus, tincidunt nec sapien ac, fermentum pellentesque orci.

Praesent semper faucibus mauris, in suscipit quam mattis et. Curabitur malesuada placerat magna. Morbi eget bibendum purus, sollicitudin faucibus enim. Donec condimentum arcu quis eros placerat sodales. Nunc non elit sodales, blandit risus vel, mattis odio. Nullam in aliquet ligula, ut viverra enim. Aliquam viverra justo non neque consequat dictum. Nunc tempor arcu sem, non blandit nisl dapibus eu. Donec vitae pretium magna. Nam risus augue, ultricies ac sollicitudin ut, rutrum eleifend purus. Donec lorem turpis, lacinia quis pretium non, sollicitudin eu enim. Curabitur eu eros congue, vehicula ante vitae, semper elit.

Fusce feugiat consectetur erat, nec venenatis massa malesuada nec. Curabitur varius convallis sem eget efficitur. Morbi congue odio non turpis cursus efficitur. In hac habitasse platea dictumst. Cras porttitor porttitor nibh id tempor. Sed bibendum quam nisl, sed congue lectus condimentum et. Curabitur nec mauris sapien. Morbi dictum ex sodales turpis ornare mattis. Suspendisse potenti. Pellentesque commodo lorem purus, id mattis nibh efficitur et. Cras tempus rutrum interdum. Vestibulum ante ipsum primis in faucibus orci luctus et ultrices posuere cubilia curae; Mauris placerat ligula eu nisl sagittis consequat. Curabitur placerat, elit quis cursus tempor, mauris augue posuere metus, ut elementum orci diam maximus lectus. Proin sed sodales nulla. Maecenas non sapien nec mauris facilisis fringilla.

\clearpage

\addcontentsline{toc}{section}{Table of Contents}

\thispagestyle{empty}
\begin{singlespace}
\renewcommand{\cftsecleader}{\cftdotfill{\cftdotsep}}
\setcounter{tocdepth}{3}
\tableofcontents
\clearpage
\addcontentsline{toc}{section}{List of Tables}
\listoftables
\clearpage
\addcontentsline{toc}{section}{List of Figures}
\listoffigures
\clearpage
\end{singlespace}

\section*{Glossary}
\addcontentsline{toc}{section}{Glossary}

\begin{tabu} to \linewidth {>{\bfseries}l>{\raggedright}X}
\toprule
Acronym A & Deffinition of A\\
Acronym B & Deffinition of B\\
Acronym C & Deffinition of C\\
\bottomrule
\end{tabu}

\clearpage

\section*{Acknowledgements}
\addcontentsline{toc}{section}{Acknowledgements}

Lorem ipsum dolor sit amet, consectetur adipiscing elit. Cras id nulla malesuada, auctor arcu at, tempor nibh. Etiam ultricies dictum nisi eu vulputate. Suspendisse tempor aliquam efficitur. Curabitur eget orci ac purus hendrerit laoreet. Proin porta arcu nisl, a fermentum tellus suscipit et. Nulla mattis orci ac tortor facilisis venenatis. Phasellus a velit tristique, tincidunt purus id, aliquet tellus. Orci varius natoque penatibus et magnis dis parturient montes, nascetur ridiculus mus. Sed et turpis sed odio faucibus aliquam nec eget est.

Phasellus ligula lacus, euismod posuere pretium sit amet, ornare condimentum nisl. Integer eu aliquet lectus, sit amet sagittis turpis. Etiam tempus bibendum nibh, eu accumsan sapien. Suspendisse tincidunt sit amet nisi vel consectetur. Donec vitae odio elit. Quisque dignissim risus feugiat nunc tempor hendrerit. Fusce consequat lectus eget felis tempus, et lobortis ante ornare. Etiam euismod quam eu scelerisque mattis. Praesent nisi lectus, tincidunt nec sapien ac, fermentum pellentesque orci.

Praesent semper faucibus mauris, in suscipit quam mattis et. Curabitur malesuada placerat magna. Morbi eget bibendum purus, sollicitudin faucibus enim. Donec condimentum arcu quis eros placerat sodales. Nunc non elit sodales, blandit risus vel, mattis odio. Nullam in aliquet ligula, ut viverra enim. Aliquam viverra justo non neque consequat dictum. Nunc tempor arcu sem, non blandit nisl dapibus eu. Donec vitae pretium magna. Nam risus augue, ultricies ac sollicitudin ut, rutrum eleifend purus. Donec lorem turpis, lacinia quis pretium non, sollicitudin eu enim. Curabitur eu eros congue, vehicula ante vitae, semper elit.

Fusce feugiat consectetur erat, nec venenatis massa malesuada nec. Curabitur varius convallis sem eget efficitur. Morbi congue odio non turpis cursus efficitur. In hac habitasse platea dictumst. Cras porttitor porttitor nibh id tempor. Sed bibendum quam nisl, sed congue lectus condimentum et. Curabitur nec mauris sapien. Morbi dictum ex sodales turpis ornare mattis. Suspendisse potenti. Pellentesque commodo lorem purus, id mattis nibh efficitur et. Cras tempus rutrum interdum. Vestibulum ante ipsum primis in faucibus orci luctus et ultrices posuere cubilia curae; Mauris placerat ligula eu nisl sagittis consequat. Curabitur placerat, elit quis cursus tempor, mauris augue posuere metus, ut elementum orci diam maximus lectus. Proin sed sodales nulla. Maecenas non sapien nec mauris facilisis fringilla.

\clearpage

\section*{Dedication}
\addcontentsline{toc}{section}{Dedication}

\begin{center}
    \vspace*{\fill}
    Insert here your dedication
    \vspace*{\fill}
\end{center}

\clearpage

\hypertarget{chapter-one-introduction}{%
\section{Chapter One: Introduction}\label{chapter-one-introduction}}

\pagenumbering{arabic}

\renewcommand{\thefigure}{1.\arabic{figure}}
\setcounter{figure}{0}
\renewcommand{\thetable}{1.\arabic{table}}
\setcounter{table}{0}
\renewcommand{\theequation}{1.\arabic{equation}}
\setcounter{equation}{0}

Accurate models of multivariate data are highly desired and of great importance in a variety of applications such as financial risk management \citep{McNeil2015}, building science and \mbox{engineering \citep{Dukhan2021, Chen2019}}, hydrology and climatology \citep{Maity2018} just to name a few. In fact, it is reasonable to assume that any natural phenomenon studied in detail usually depends on more than one factor, and nature is multivariate \mbox{\citep[Ch. 2]{Bohm2013}}.

The multivariate normal distribution serves as a fundamental building block in classical multivariate statistics due to its mathematical tractability, but is constrained by elliptical symmetry and light-tailed behavior rendering it an inappropriate model for datasets exhibiting asymmetry or heavy-tailed behavior. Many datasets in fields such as insurance, finance, data networks, econometrics, psychometrics, and climatology exhibit asymmetry and/or heavy-tailed behavior \mbox{\citep{Genton2004, Resnick2007}} so that extensions of the multivariate normal distribution are desired to model characteristics common in real world data.

Elliptically-contoured or elliptical distributions introduced by \mbox{\citet{Kelker1970}} and discussed in detail by \mbox{\citet{Fang1990}} generalize the multivariate normal to allow for both heavy and light-tailed behaviors. The class of elliptical distributions includes distribution families characterized by density level sets that are ellipsoids such as the multivariate normal, multivariate \(t\), multivariate logistic, and multivariate exponential power families. Elliptical distributions are commonly used to model financial data such as stock returns and risk \mbox{\citep{McNeil2015}} where the elliptical symmetry constraint might be reasonable.

The \ac{pdf} of a random vector \(\Xb \in \R^p\) with an elliptical distribution can be written as \eeq[elliptical-pdf]{
  \fnc{f}{\xb} = \fnc{f_0}{\fnc{\de}{\xb; \mub, \Sigma}; \gbm},
} where \mbox{$\Sigma \in \R^{p\times p}$} is a positive-definite matrix determining the shape of the ellipsoidal density level sets, \(\mub \in \R^p\) is the location parameter and \({f_0:[0, \nf)\to(0, \nf)}\) is a decreasing function called the density generator with finite parameter vector \(\gbm\) that controls the decay of the density.

We define density level sets for a continuous density \(f\) as \eeq[density-level-set]{
  \set{f > c} \coloneqq \set{\xb \in \R^p : \fnc{f}{\xb} > c} = s_c D, \qquad 0 < c < c_0 \coloneqq \sup f,
} where \(s_c\) is a positive real number corresponding to \(\fnc{f}{\xb} = c\), and the set \(D\) is called the shape set. For elliptical distributions, the shape set is given by \eqq{
  D = \set{\xb \in \R^p : \fnc{\de}{\xb; \mub, \Sigma} < 1}.
}

By rewriting the density of an elliptical distribution as in Equation \nolinebreak\eqref{Eq:elliptical-pdf}, it is clear that each point on the boundary of a density level set, \(\set{\xb \in \R^p : \fnc{\de}{\xb; \mub, \Sigma} = s_c}\), produces the same density value, \(\fnc{f}{\xb} = c\). As a result, the \ac{pdf} of an elliptical distribution is completely determined by \(\Sigma\) which analytically describes the shape of the density level sets, \(\mub\) which gives the location of the center of the density level sets, and \(s_c\) which specifies a density level set by its ``distance'' from \(\mub\). Thus, density level sets are scaled copies of the shape set.

Using the transformation \(\de:\R^p\to\R\), Equation \eqref{Eq:elliptical-pdf} demonstrates how the multivariate elliptical density can be rewritten as a density for the random variable \(\fnc{\de}{\Xb; \mub, \Sigma}\). As examples, we use Equation \eqref{Eq:elliptical-pdf} to rewrite the bivariate normal and bivariate \(t\) densities. Suppose \(\mub \in \R^p\) and \(\Sigma \in \R^{p\times p}\) are given. The bivariate normal \ac{pdf} can be written as \begin{align*}
  \fnc{f}{\xb; \mub, \Sigma} &= k_1\expo{-\frac{1}{2}\paren{\xb - \mub}\tr\Sigma^{-1}\paren{\yb - \mub}}
    = k_1\expo{-\frac{\fnc{\de}{\xb; \mub, \Sigma}^2}{2}} 
    = \fnc{f_{01}}{\fnc{\de}{\xb; \mub, \Sigma}},
  \end{align*}
for normalizing constant \(k_1\) and density generator \(f_{01}\). Similarly, the bivariate \(t\) \ac{pdf} can be written as \begin{align*}
  \fnc{f}{\xb; \mub, \Sigma, \nu} &= k_2\brac{1 + \frac{1}{\nu}\paren{\xb - \mub}\tr\Sigma^{-1}\paren{\xb - \mub}}^{-\frac{\nu + 2}{2}}
    = k_2\brac{1 + \frac{\fnc{\de}{\xb; \mub, \Sigma}^2}{\nu}}^{-\frac{\nu + 2}{2}} 
    = \fnc{f_{02}}{\fnc{\de}{\xb; \mub, \Sigma}},
  \end{align*}
for normalizing constant \(k_2\) and density generator \(f_{02}\).

For an elliptically distributed random vector \(\Xb\), consider the transformations \eqq{
  R = \fnc{\de}{\Xb; \mub, \Sigma} \qquad \text{and} \qquad \T = \frac{\Xb}{R},
} where \(R\) is the Mahalanobis distance of \(\Xb\) from \(\mub\) and \(\T\) is its direction. Random variables \(R\) and \(\T\) are independent \mbox{\citep{Kamiya2008}}. For a fixed \(\Sigma\), the distribution of \(\T\) is the same for all elliptically contoured distributions, so we can construct arbitrary elliptical distributions by changing the distribution of \(R\). This property, often referred to as null robustness \citep{Kariya1989}, states that distributional results concerning \(\T\) derived under the assumption of normality continue to hold for all elliptical distributions having the same \mbox{$\Sigma$ \citep{Kamiya2008}}. We can thus view density estimation for an elliptical distribution as a two-step procedure that estimates the independent distributions of \(R\) and \(\T\).

Unfortunately, the elliptical distributions are too narrow in scope as they only differ from the multivariate normal distribution through the distribution of \(R\) and cannot model data exhibiting skewness or asymmetry due to the elliptical symmetry constraint. Scatterplots of multivariate data frequently display shapes of contour lines/surfaces that are not ellipses/ellipsoids \citep{Liebscher2020} and despite the popularity of elliptical distributions for modelling financial data, stock returns are not necessarily \mbox{elliptical \citep{Chicheportiche2012}}. Hence, a generalization of elliptical distributions is needed.

As shown in \citet{Kamiya2008}, the independence of distance and direction as well as null robustness continue to hold beyond the elliptical distributions so long as the notion of distance is defined properly. The need for flexibility beyond the elliptical distributions and the properties of independence of distance and direction and null robustness motivate the development of the class of star-shaped distributions whose densities have level sets that are scaled copies of some arbitrary star-shaped set.

Star-shaped distributions \citep{Kamiya2008} comprise a highly flexbile class of distributions that generalize the elliptical class by removing the elliptical symmetry constraint and allow for density level sets to be scaled copies of any star-shaped set. A set \(D\) is star-shaped with respect to \(\mub\) if \(\xb\in D\) implies \(\xi\xb + \paren{1-\xi}\mub \in D\) for all \(0 < \xi < 1\). This class was previously studied as the \(\nu\)-spherical class \citep{Fernandez1995} and the directionally dispersed class (\(D\)-class) \citep{Ferreira2005}.

In this thesis, we focus on a sub-class of star-shaped distributions characterized by having continuous density functions with homothetic level sets. These densities are referred to as homothetic \mbox{densities \citep{Balkema2010}} and have density level sets that are scaled copies of some star-shaped set \(D\) with respect to a center \(\mub\) analytically described by \mbox{function $n_D:\R^p \to [0, \nf)$} called a gauge function that is homogeneous of degree \(1\) \mbox{and $D = \set{\xb \in \R^p | \fnc{n_D}{\xb - \mub} < 1}$} is an open, bounded set. In addition, we assume that for each vector \mbox{$\paren{\xb-\mub} \in D$}, there is one positive real \(r\) such \mbox{that $r\cdot\paren{\xb-\mub}$} lies on the boundary of \(D\). This assures a continuous gauge function on the open cone \(D_{\nf} = \bigcup_n nD\).

A density \(f\) on \(\R^p\) is said to belong to the class of homothetic densities, denoted \(f \in \HC{D}\), if the shape set \(D\) is a bounded open star-shaped set in \(\R^p\) and if \(f\) has the following representation, \eeq[star-pdf]{
  \fnc{f}{\xb} = \fnc{f_0}{\fnc{n_D}{\xb - \mub}},
} where \(f_0:[0, \nf) \to (0, \nf)\) is a decreasing continuous function called the density generator. Densities for the elliptical distributions belong to \(\HC{D}\) with gauge function \(\fnc{n_D}{\xb - \mub} = \fnc{\de}{\xb; \mub, \Sigma}\).

For random vector \(\Xb \in \R^p\) with \ac{pdf} \(f \in \HC{D}\), we use the following transformations of \(\Xb\) to create variables for distance and direction, respectively, \eqq{
  R = \fnc{n_D}{\Xb - \mub} \qquad \text{and} \qquad \T = \frac{\Xb}{R},
} where \(R\) is the star-generalized radius variable \citep{Liebscher2016} corresponding to \(\Xb\) and \(\T\) is its direction. Homothetic densities thus generalize the elliptical distributions through these notions of distance and direction that preserve independence of \(R\) and \(\T\) as well as null robustness on the distribution of \(\T\).

There are three main goals of this research. The first is to review, implement, and critique various approaches for estimating distributions from multivariate datasets assumed to have homothetic densities. We start with a focus on \(\R^2\) since methods that perform poorly in \(\R^2\) are unlikely to generalize to higher dimensions. We also propose a parametric subclass of distributions with homothetic densities called trochoidal distributions that generalize the elliptical class. The second is to analyze the asymptotic behavior of multivariate risk measures popular in finance and insurance when risk factors have a joint \ac{pdf} \(f\in\HC{D}\) with a light-tailed density generator. Finally, we plan to investigate the estimation of high-risk regions for homothetic densities usually requiring extrapolation beyond the observed data.

The remainder of this report is outlined as follows: two major research objectives are discussed in detail in Sections \ref{SS:intro-modelling} and \ref{SS:intro-financial-risk}, respectively. A third major objective has not yet been started and is included in a list of research objectives to be completed at the end of the report. Section \ref{S:background} provides definitions, results, and properties that facilitate a formal discussion of the class of homothetic densities and motivate approaches for estimation. Section \ref{S:shape-set} discusses nonparametric and semiparametric approaches for estimating the shape set and presents preliminary results on simulated data. Section \ref{SS:trochoidal-dist} introduces the class of trochoidal distributions, comprised of fully parametric families, and performs \ac{ml} estimation to fit candidate families to simulated data. Finally, Section \ref{S:objectives} discusses remaining work to be completed.

\hypertarget{SS:intro-modelling}{%
\subsection{Modelling and Estimation}\label{SS:intro-modelling}}

Given a dataset generated from a distribution with \ac{pdf} \(f\in\HC{D}\), of primary interest is the estimation of \(f\) and performing inference on the population's parameters. In general, estimation of \(f \in \HC{D}\) involves estimation of \(\mub\), \(D\) and \(f_0\). For a dataset \(\xb_1, \dots, \xb_n\), an intuitive estimation approach might be to estimate the shape set and its gauge function, \(\fncDtest[\est{D}]{n}{\xb - \mub}\), producing estimates of the star-generalized radius variables \(\est{r}_1, \dots, \est{r}_n\) which can then be used to estimate the density generator. This process may be repeated using the estimate of the density generator to update the estimate of the shape set and so on until convergence. We call this general strategy the ``iterative approach'' as it is similar to the EM-Algorithm \citep{Dempster1977}, but such a strategy does not appear in the literature.

The framework for shape set estimation presented in \citet{Ferreira2005} uses free-knot splines with reversible-jump Markov chain Monte Carlo to produce samples from the posterior distribution of \(D\). While this leads directly to inference, it can be computationally intensive, require many inputs from the user such as prior distributions and tuning parameters, and assumes knowledge of the density generator.

Parametric estimators for the shape set are presented in \citet{Liebscher2016} along with both parametric and non-parametric estimators for \(f_0\). By combining estimators of \(D\) and \(f_0\), both fully and semiparametric estimators for the entire density are derived. This approach is constrained by requiring \(D\) to have a parametric form symmetric about the origin with \(\Exp{R^2} < \nf\).

The nonparametric shape set estimator of \citet{Kamiya2019} applies a transformation to the kernel density estimator of direction and is very promising in that it is strongly consistent with respect to the Hausdorff distance and is not affected by the density generator. This latter fact implies that one can estimate the shape set first, though inference is affected by the number of observations, the kernel bandwidth selection, and dimensionality. The authors do not discuss the use of this estimator or its performance as part of the larger density estimation problem.

Flexible parametric families with few, interpretable parameters are desireable for density estimation and inference as they allow for estimation via maximum likelihood. Under regularity conditions \citep[Section 6.1]{Hogg2005}, the maximum likelihood estimators are asymptotically normally distributed and inference follows directly from the observed information matrix and deviance function. This motivates the search for such parametric families. Fully parametric model classes are introduced in \citet{Liebscher2020} enabling the maximum likelihood approach to estimating \(f \in \HC{D}\), but are separate classes of distributions.

We introduce a fully parametric subclass of star-shaped distributions on \(\R^2\) with densities \(f \in \HC{D}\) called the trochoidal distributions, characterized by having density level sets whose boundaries can be traced by centered trochoids. This class is the union of the epitrochoidal and hypotrochoidal distribution families also introduced here. In the bivariate setting, this class offers flexible distributions with only a few parameters whose interpretations are direct and practical. In addition, the trochoidal distributions are a generalization of the elliptical class since an ellipse is a trochoid \citep{Pedoe1975}. This generalization is geometric in the sense that the shape of the density contours can be traced by any centered trochoid under certain constraints on the trochoidal parameters that lead to identifiable distributions with densities \(f \in \HC{D}\). The trochoidal distributions can be estimated via maximum likelihood and inference follows as in \citet{Liebscher2020}. The parameters have straightforward interpretations regarding the degree, direction, and number of directions of dependence.

\clearpage

\hypertarget{chapter-two-insert-title-chapter-here}{%
\section{Chapter Two: Insert Title Chapter Here}\label{chapter-two-insert-title-chapter-here}}

\renewcommand{\thefigure}{2.\arabic{figure}}
\setcounter{figure}{0}
\renewcommand{\thetable}{2.\arabic{table}}
\setcounter{table}{0}
\renewcommand{\theequation}{2.\arabic{equation}}
\setcounter{equation}{0}

\clearpage

\hypertarget{chapter-three-insert-title-chapter-here}{%
\section{Chapter Three: Insert Title Chapter Here}\label{chapter-three-insert-title-chapter-here}}

\renewcommand{\thefigure}{3.\arabic{figure}}
\setcounter{figure}{0}
\renewcommand{\thetable}{3.\arabic{table}}
\setcounter{table}{0}
\renewcommand{\theequation}{3.\arabic{equation}}
\setcounter{equation}{0}

\clearpage

\hypertarget{chapter-four-insert-title-chapter-here}{%
\section{Chapter Four: Insert Title Chapter Here}\label{chapter-four-insert-title-chapter-here}}

\renewcommand{\thefigure}{4.\arabic{figure}}
\setcounter{figure}{0}
\renewcommand{\thetable}{4.\arabic{table}}
\setcounter{table}{0}
\renewcommand{\theequation}{4.\arabic{equation}}
\setcounter{equation}{0}

\clearpage

\hypertarget{chapter-five-insert-title-chapter-here}{%
\section{Chapter Five: Insert Title Chapter Here}\label{chapter-five-insert-title-chapter-here}}

\renewcommand{\thefigure}{5.\arabic{figure}}
\setcounter{figure}{0}
\renewcommand{\thetable}{5.\arabic{table}}
\setcounter{table}{0}
\renewcommand{\theequation}{5.\arabic{equation}}
\setcounter{equation}{0}

\clearpage

\hypertarget{synthesis-and-conclusion}{%
\section{Synthesis and Conclusion}\label{synthesis-and-conclusion}}

Lorem ipsum dolor sit amet, consectetur adipiscing elit. Cras id nulla malesuada, auctor arcu at, tempor nibh. Etiam ultricies dictum nisi eu vulputate. Suspendisse tempor aliquam efficitur. Curabitur eget orci ac purus hendrerit laoreet. Proin porta arcu nisl, a fermentum tellus suscipit et. Nulla mattis orci ac tortor facilisis venenatis. Phasellus a velit tristique, tincidunt purus id, aliquet tellus. Orci varius natoque penatibus et magnis dis parturient montes, nascetur ridiculus mus. Sed et turpis sed odio faucibus aliquam nec eget est.

Phasellus ligula lacus, euismod posuere pretium sit amet, ornare condimentum nisl. Integer eu aliquet lectus, sit amet sagittis turpis. Etiam tempus bibendum nibh, eu accumsan sapien. Suspendisse tincidunt sit amet nisi vel consectetur. Donec vitae odio elit. Quisque dignissim risus feugiat nunc tempor hendrerit. Fusce consequat lectus eget felis tempus, et lobortis ante ornare. Etiam euismod quam eu scelerisque mattis. Praesent nisi lectus, tincidunt nec sapien ac, fermentum pellentesque orci.

Praesent semper faucibus mauris, in suscipit quam mattis et. Curabitur malesuada placerat magna. Morbi eget bibendum purus, sollicitudin faucibus enim. Donec condimentum arcu quis eros placerat sodales. Nunc non elit sodales, blandit risus vel, mattis odio. Nullam in aliquet ligula, ut viverra enim. Aliquam viverra justo non neque consequat dictum. Nunc tempor arcu sem, non blandit nisl dapibus eu. Donec vitae pretium magna. Nam risus augue, ultricies ac sollicitudin ut, rutrum eleifend purus. Donec lorem turpis, lacinia quis pretium non, sollicitudin eu enim. Curabitur eu eros congue, vehicula ante vitae, semper elit.

Fusce feugiat consectetur erat, nec venenatis massa malesuada nec. Curabitur varius convallis sem eget efficitur. Morbi congue odio non turpis cursus efficitur. In hac habitasse platea dictumst. Cras porttitor porttitor nibh id tempor. Sed bibendum quam nisl, sed congue lectus condimentum et. Curabitur nec mauris sapien. Morbi dictum ex sodales turpis ornare mattis. Suspendisse potenti. Pellentesque commodo lorem purus, id mattis nibh efficitur et. Cras tempus rutrum interdum. Vestibulum ante ipsum primis in faucibus orci luctus et ultrices posuere cubilia curae; Mauris placerat ligula eu nisl sagittis consequat. Curabitur placerat, elit quis cursus tempor, mauris augue posuere metus, ut elementum orci diam maximus lectus. Proin sed sodales nulla. Maecenas non sapien nec mauris facilisis fringilla.

\clearpage

\section*{Bibliography}
\addcontentsline{toc}{section}{Bibliography}

\noindent
\leftskip=2em
\parindent=-2em

\hypertarget{refs}{}

\clearpage

\setlength{\parindent}{4em} 
\linespread{1}
\doublespacing

\section*{Appendices}
\addcontentsline{toc}{section}{Appendices}

\section*{Appendix A - Supplementary information for "Name of Chapter Two"}
\addcontentsline{toc}{section}{Appendix A}

\renewcommand{\thefigure}{A2.\arabic{figure}}
\setcounter{figure}{0}
\renewcommand{\thetable}{A2.\arabic{table}}
\setcounter{table}{0}
\renewcommand{\theequation}{A2.\arabic{equation}}
\setcounter{equation}{0}

\section*{Appendix B - Supplementary information for "Name of Chapter Three"}
\addcontentsline{toc}{section}{Appendix B}

\renewcommand{\thefigure}{A3.\arabic{figure}}
\setcounter{figure}{0}
\renewcommand{\thetable}{A3.\arabic{table}}
\setcounter{table}{0}
\renewcommand{\theequation}{A3.\arabic{equation}}
\setcounter{equation}{0}

\section*{Appendix C - Supplementary information for "Name of Chapter Four"}
\addcontentsline{toc}{section}{Appendix C}

\renewcommand{\thefigure}{A4.\arabic{figure}}
\setcounter{figure}{0}
\renewcommand{\thetable}{A4.\arabic{table}}
\setcounter{table}{0}
\renewcommand{\theequation}{A4.\arabic{equation}}
\setcounter{equation}{0}

\end{document}
